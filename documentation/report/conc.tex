\section*{Conclusion \& Future Work}
In conclusion the project team can be proud of the results of the project. Although we were held up by many data collection and data quality issues we managed to experimentally construct a pipeline from collection tweets to time series of tweet indicators and price data collected online to cleaned an interpolated time series. The neural networks and incorporation of additional data sources did not deliver the predictive power hoped for yet once trained provide very accurate day-to-day predictions without having to be retrained. Due to the many issues we encountered with collecting, setting up the infrastructure and analyzing tweets we were not able to collect enough to data to evaluate the use of twitter indicators. However, since we have successfully set up the infrastructure and filtering of tweets we are positive that this can be done in future work.\\
In order to refine the prediction models used and to complete the pipeline it might a good idea to reapply this approach to the \href{http://www.kemendag.go.id/en/economic-profile/prices/national-price-table}{Indonesian national price table} that has recently been published by the Indonesian Ministry of Trade. Twitter statistics very much favor Indonesia with 11.7\% of the population using twitter over India where the the percentage only amounts to 1.3\%.\\
Taking the idea of twitter indicators further our map visualization shows that the volume of tweets in India has massively increased in the last few years. By continuing the collection of tweets from India the tweet indicators could be bolstered and compared to price time series. If significant relationships are found the tweet indicators could be used to support real-time prediction of prices when prices monitored by the government are not yet available.\\
Another interesting point would be to take the analyze of the social media content further order to figure if it is possible to track the patterns of food supply in throughout the country. Such a system could help the highly decentralized Indian government in managing the allocation of resources including food market interventions.

\section*{Closing remarks}
The whole project team would like to sincerely thank Professor Christoph Koch for giving us the possibility to propose this project and implement it despite the uncertainty of its outcome. We would further like to thank our TA Aleksandar Vitorovic for his valuable criticism, advice and motivation.
