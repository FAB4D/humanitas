\section*{Introduction}
\subsection*{Motivation}
Many countries in the global south are affected by high and especially volatile prices of staple foods and other commodities. These circumstances most heavily impact low-income consumers, small producers and food traders. These people in general have no direct access to real-time price information, let alone price predictions, that would allow them to plan ahead and thereby at least mitigate the impacts of price volatility. Furthermore many governments don’t have sophisticated models to coordinate their interventions on the commodity markets and optimally distribute their commodities.\\

\subsection*{Project goal}
The main idea of this project was to find indicators present in social media that help in monitoring and predicting prices of basic commodities. In this context we wanted to conceive a commodity price and supply prediction framework for developing countries tough combining commodity price data from official sources (governments, international organizations such as the World Bank and the IMF) and indicators derived from social media data. We set out piloting this approach for a specific country with freely available price data, India, and hoped to achieve the prediction of commodity prices on a daily basis at state level. Due to the limited amount of penetration of twitter in India and the nature of discussions on twitter we expected filtering out a significant amount of tweets relevant to our needs a hard task. For further details we refer to our project proposal.