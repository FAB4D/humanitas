\section*{Visualization of results}

We essentially represent the results of data collection and prediction on the Indian map. 

\subsection*{Geo-visualization}

\subsubsection*{Technology}
LeafletJS: A javascript framework that builds the world map on top of OpenStreetMap library. MapboxJS is also used to extend the functionality of different behaviors on the map as well as enhance the UI design.

\subsubsection*{Content}
There are several layers shown on the map:\\
\begin{itemize}
\item Tweets density of users in different Indian states.
\item Price comparison between states\\
\end{itemize}

\subsection*{Atomic visualization}
\subsubsection*{Technology}
HighchartsJS

\subsubsection*{Tweets per state}


\subsubsection*{Price fluctuation}

\subsection*{Results from Visualization}
\emph{Fabian Brix}
Our visualization of tweets per region exemplifies the result of \href{http://web.worldbank.org/WBSITE/EXTERNAL/TOPICS/EXTPOVERTY/EXTPA/0,,contentMDK:20208963~menuPK:435735~pagePK:148956~piPK:216618~theSitePK:430367,00.html}{poverty in India's most populous state of Uttar Pradesh} (200 million inhabitants) in the use of social media and probably Information \& Communication Technologies (ICTs) in general. When moving through time in the visualization one can see that the states Karnataka (Bangalore), Maharashtra (Mumbai) ICT of Delhi are constantly ahead of Uttar Pradesh in the number of tweets they produce although their population is significantly smaller. Another phenomenon that can be observed is that Maharasthra and Karnataka are having a head-to-head race starting in 2007 although the population of Karnataka is only half the size of that of Maharashtra. This can be attributed to the fact that the state capital Bangalore is the major IT hub of India and therefore boasts more early adopters. In 2010 as more and more people take to twitter Maharashtra raises far ahead in the number of tweets produced.
